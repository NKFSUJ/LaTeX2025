\documentclass[border=1cm]{standalone}
\usepackage{graphicx}
\usepackage{tikz}
\usepackage[T1]{fontenc}


\begin{document}


 \begin{tikzpicture}
%  \node at (0,0) {\includegraphics[width=10cm]{images/2025-11-17-17-21-58.png}};


%bułka
\draw[line width=0.1cm, black, 
fill=brown!50!white,
rounded corners=0.2cm] (0,0.8) -- (0.8,0.8) .. controls++(0.5,-0.1) .. (1.9,0) .. controls++(0.3,-0.6) .. (2,-1.65) .. controls++(-0.5,-0.5) .. (1,-2.3) .. controls++(-0.5,-0.1) .. (0,-2.4) .. controls++(-1.4,0.1) .. (-1.8,-2) .. controls ++(-0.4,0.5) .. (-2.3,-1) .. controls++(0.0,0.5) .. (-2.1,0) .. controls++(0.8,0.7) .. (0,0.8);

%oko lewe
\draw[thick, white, 
fill=black,
rounded corners=0.1cm] (0,-1) --++ (.2,0) --++ (.15,-.1) --++ (.1,-.1) --++ (.1,-.2) --++ (0,-.2) --++ (-.15,-.15) --++ (-.15,0) --++ (-.1,.05) --++ (-.2,0) --++ (-.2,-.05) --++ (-.15,0) --++ (-.1,.15) --++ (0,.15) --++ (0.05,0.15) --++ (0.15,0.15) --++ (0.1,0.1) --++ (.2,.05);
\draw[fill=white] (.22,-1.22) circle (.12);
\draw[fill=white] (.4,-1.35) circle (.06);

%oko prawe
\draw[thick, white, 
fill=black, 
rounded corners=0.1cm, rotate around={10:(1.2,-1.2)}] (1.2,-0.8) --++ (.2,0) --++ (.15,-.1) --++ (.1,-.1) --++ (.1,-.2) --++ (0,-.2) --++ (-.15,-.15) --++ (-.15,0) --++ (-.1,.05) --++ (-.2,0) --++ (-.2,-.05) --++ (-.15,0) --++ (-.1,.15) --++ (0,.15) --++ (0.05,0.15) --++ (0.15,0.15) --++ (0.1,0.1) --++ (.2,.05);
\draw[fill=white] (1.42,-1.02) circle (.12);
\draw[fill=white] (1.6,-1.15) circle (.06);

%włosy
\draw[line width=0.15cm, brown!80!white] (-.1,.1) arc(60:120:1.3);
\draw[line width=0.15cm, brown!80!white] (-.1,.1) arc(90:145:1.3);
\draw[line width=0.15cm, brown!80!white] (-.1,.1) arc(90:20:0.9);
\draw[line width=0.15cm, brown!80!white] (-.1,.1) arc(120:60:1.3);
\draw[line width=0.15cm, brown!80!white] (-.1,.1) arc(0:40:0.7);

%napisy
\node at (0,2) {\Large jak nazywa się moment};
\node at (0,1.4) {\Large przed zjedzeniem bułki?};
\node at (0,-3) {\Large preambuła};
\node at (0,-3.6) {\Large \color{black!10!white} ha ha ha ha};


% siatka nie ruszać

% \draw[gray!50, thin, step=0.1cm] (-5,-3) grid (5,3);
% \draw[gray, thin, step=1cm] (-5,-3) grid (5,3);
% \draw[black, thin, step=5cm] (-5,-3) grid (5,3);

\end{tikzpicture}

\end{document}