\documentclass[25pt,a2paper]{tikzposter}
\usepackage[utf8]{inputenc}
\usepackage[T1]{fontenc}
\usepackage{polski}
\usepackage{amsmath, amssymb, amsfonts}
\usepackage{braket}
\usepackage{tikzducks}
\usepackage{comment}
\usepackage{pgfplots}
\usepackage{multicol}
\usepackage{qrcode}
\usepackage{xcolor}
\usepackage{moresize}
\usepackage{microtype}

\usepackage{enumitem}
\setlist[1]{itemsep=-4pt}

% \usepackage{polyglossia}
% \setmainlanguage{polish}

% \usepgfplotslibrary{external} 
% \tikzexternalize


\usetheme{Board}
% \useblockstyle[titleinnersep=16pt]{Board}
\colorlet{framecolor}{purple}
\usetitlestyle{VerticalShading}
% \background{violet!10}
% \usecolorpalette{GreenGrayViolet}

% \defineblockstyle{opacitystyle}{
%     \tikzset{
%         draw=none,
%         fill=blockbodybg.color,
%         fill opacity=0.3
%     }
% }
% \useblockstyle{opacitystyle}
% \defineblockstyle{PrzezroczystyBlok}{
%     titlewidthscale=0.9,
%     bodywidthscale=1,
%     titleleft,
%     titleoffsetx=0pt,
%     titleoffsety=0pt,
%     bodyoffsetx=0pt,
%     bodyoffsety=0pt,
%     bodyverticalshift=0pt,
%     roundedcorners=0,
%     titleinnersep=6mm,
%     bodyinnersep=1cm
% }{
%     \draw[draw=none, fill=blockbodybg.color, fill opacity=0.7] % 0.7 = 70% nieprzezroczystości
%     (blockbody.south west) rectangle (blockbody.north east);
% }
% \useblockstyle{PrzezroczystyBlok}
% \defineblockstyle{sampleblockstyle}{
% titlewidthscale=0.9, bodywidthscale=1,titleleft,
% titleoffsetx=0pt, titleoffsety=0pt, bodyoffsetx=0mm, bodyoffsety=15mm,
% bodyverticalshift=10mm, roundedcorners=5, linewidth=2pt,
% titleinnersep=6mm, bodyinnersep=1cm
% }{
% \draw[color=framecolor, fill=blockbodybgcolor,
% rounded corners=\blockroundedcorners] (blockbody.south west)
% rectangle (blockbody.north east);
% \ifBlockHasTitle
% \draw[color=framecolor, fill=blocktitlebgcolor,
% rounded corners=\blockroundedcorners] (blocktitle.south west)
% rectangle (blocktitle.north east);
% \fi
% }


% \draw[inner sep=0pt, line width=0pt, color=red, fill=backgroundcolor!30!black]
%(bottomleft) rectangle (topright);

\defineblockstyle{TornOut}{
    titlewidthscale=1, bodywidthscale=1, titlecenter,
    titleoffsetx=0pt, titleoffsety=0pt, bodyoffsetx=0pt, bodyoffsety=0pt,
    bodyverticalshift=-1.2cm, roundedcorners=0, linewidth=1.2pt,
    titleinnersep=1cm, bodyinnersep=1cm 
}{
    \ifBlockHasTitle%
        \coordinate (topright) at (blocktitle.north east);
    \else
        \coordinate (topright) at (blockbody.north east);
    \fi%
    \draw[color=blocktitlebgcolor, fill=blockbodybgcolor,%
        line width=\blocklinewidth, drop shadow={shadow xshift=0.2cm, shadow yshift=-0.2cm,opacity=0.3}, %
        decorate, decoration={random steps,segment length=1.5cm,amplitude=0.15cm},
        % decorate, decoration={penciline,amplitude=0.2cm}
        fill opacity = 0.85
    ] (blockbody.south west) rectangle (topright);%
}



\definebackgroundstyle{samplebackgroundstyle}{
    \tikzset{
        shift={(-0.5\textwidth,-0.5\textheight)},
        scale=2.4,
    }
    \draw[line width=0pt, bottom color=backgroundcolor, top color=backgroundcolor] (bottomleft) rectangle (topright); 
    \newcommand{\xsize}{7}
    \newcommand{\ysize}{8}
    \foreach \i in {1,...,56}
    {
        \pgfmathtruncatemacro{\y}{(\i - 1) / \xsize}
        \pgfmathtruncatemacro{\x}{\i - \xsize * \y - 1}
        % \pgfmathtruncatemacro{\label}{\x + \xsize * (\xsize - 1 - \y)};
        \randuck[xshift=2.5*\x cm, yshift=2.5*\y cm]
    }
}
\usebackgroundstyle{samplebackgroundstyle}
\usecolorstyle[colorOne=violet]{Sweden}
\colorlet{backgroundcolor}{violet!70!white}

\title{{\centering \scalebox{1.5}{Warsztaty \LaTeX{}a}}}
\author{\Large $\bra{N}$aukowe $\hat{K}$oło $\ket{F}$izyków \& {Koło Matematyczno--Przyrodnicze}}
\date{\today}
\institute{{\large \centering Wydział Fizyki, Astronomii i Informatyki Stosowanej UJ}}

\linespread{0.9}

\begin{document}

\colorlet{titlebgcolor}{violet!30!white}
\colorlet{framecolor}{violet!30!white}
\maketitle

% \begin{comment}
\begin{columns}

\column{0.1}
\column{0.8}

% \colorlet{blockbodybgcolor}{{\color{green}\pgfsetfillopacity{0.5}}}

\block{Co to \LaTeX?}{
    Latex to taki śmieszny system do składania tekstu\\
% }
    % \begin{subcolumns}
    % \subcolumn{0.8}
    % \block{Zalety}{
    %     \begin{itemize}
    %         \raggedright
    %         \begin{multicols}{2}
    %         \item wygląda profesjonalnie
    %         \item wspiera przypisy, bibliografię i~spisy treści
    %         \item wspiera skomplikowane wzory matematyczne
    %         \item jest darmowy
    %         \end{multicols}
    %     \end{itemize}
    % }
    % \subcolumn{0.2}
    % \block{Wady}{\centering brak\vspace{50pt}}
    % \end{subcolumns}
% }
    % \begin{subcolumns}
    % \subcolumn{0.8}
    % \vfill
    \begin{minipage}[t]{0.5\textwidth}
        \colorlet{innerblocktitlebgcolor}{green!70!black}
    \innerblock{Zalety}{\normalsize
        \begin{itemize}
            \raggedright
            \begin{multicols}{2}
            \item wygląda profesjonalnie
            \item wspiera przypisy, bibliografię i~spisy treści
            \item wspiera skomplikowane wzory matematyczne
            \item jest darmowy
            \end{multicols}
        \end{itemize}
    }
    \end{minipage}
    \hfill
    % \subcolumn{0.2}
    \begin{minipage}[t]{0.15\textwidth}
        \colorlet{innerblocktitlebgcolor}{red!70!black}
    \innerblock{Wady}{\normalsize \centering \textbullet\ brak\vspace{70pt}}
    \end{minipage}
    % \vfill
    % \end{subcolumns}
    }
% \block{Czy \LaTeX{} jest lepszy niż Word?}{TAK}


% \block{Co się stanie jak będę pisał w Wordzie?}{
% %\begin{tikzfigure}
% \centering
% \includegraphics[width=0.4\linewidth]{ktopisze.jpg}
% %\end{tikzfigure}
% }

\block{Kiedy i gdzie?}{
\begin{minipage}[t]{0.65\linewidth}
\begin{itemize}%[$\leftarrow$]
\item \textbf{5--6 grudnia 2025}
\item Wydział Fizyki, Astronomii i Informatyki Stosowanej UJ, ul.~Łojasiewicza 11
\item zapisy do ...
\end{itemize}
\end{minipage}
\begin{minipage}[t]{0.25\linewidth}
    % \innerblock{}{
    \centering
    \duck[santa, beard=white!80!brown,
        xshift=3cm, scale=3, yshift=0cm]
    % }
\end{minipage}
}

\block{Program}{
\begin{minipage}[t]{0.5\linewidth}
    {\centering \bf Piątek, 5 grudnia\par}
    \begin{itemize}
        \item wprowadzenie
        \item podstawy
        \item tryb matematyczny
        \item tabele
        \item grafiki
    \end{itemize}
\end{minipage}
\begin{minipage}[t]{0.5\linewidth}
    {\centering \bf Sobota, 6 grudnia\par}
    \begin{itemize}
        \item bibliografia
        \item prezentacje
        \item tikz
        \item plakaty
        \item może coś jeszcze
    \end{itemize}
\end{minipage}
% \end{columns}
}

% \block{Będą kaczki?}{
% \centering
% \begin{tikzpicture}
% \duck[scale=2.2,speech={Będą},
% bubblecolour=cyan!20!white,
% laughing]
% \end{tikzpicture}
% }

\begin{subcolumns}
    \subcolumn{0.3}
    \subcolumn{0.35}
    \block{więcej info}{
        \centering
        \qrcode[height=5cm]{https://www.youtube.com/watch?v=dQw4w9WgXcQ}
    }
    \subcolumn{0.35}
    \block{zapisy}{
        \centering
        \qrcode[height=5cm]{https://www.youtube.com/watch?v=cErgMJSgpv0}
    }
\end{subcolumns}

\column{0.1}

\end{columns}
% \end{comment}

\end{document}